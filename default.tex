\documentclass[12pt]{article}
% Opciones {{{
\usepackage[pdfa, pdfusetitle, unicode=true]{hyperref}
\usepackage[spanish]{babel}
\usepackage[margin=1.5cm, a4paper]{geometry}
\usepackage{luacode}
\usepackage[shortlabels]{enumitem}
\usepackage{import}
\usepackage{xcolor}
\usepackage{fontspec}

% Cosas para la bibliografía {{{
\usepackage{csquotes}
\usepackage[natbib,
	backend=biber,
	style=apa,
	apabackref=true
	]{biblatex}
\addbibresource{default.bib}
% }}}

\setmonofont{InconsolataGo Nerd Font}

\newcommand{\btw}{{\color{arch}\texttt{ }} }

%Esto sirve para poner imágenes{{{
\usepackage{graphicx}
\usepackage{svg}
\usepackage{subcaption}

\usepackage{float}
\usepackage{pgfplots}
\usepackage{cancel}

\pgfplotsset{compat=1.16}
\graphicspath{ {ima/} }
%}}}
%Colores de los links {{{
\definecolor{red}{HTML}{F22C40}
\definecolor{green}{HTML}{5AB738}
\definecolor{yellow}{HTML}{D5911A}
\definecolor{blue}{HTML}{407EE7}
\definecolor{magenta}{HTML}{6666EA}
\definecolor{cyan}{HTML}{00AD9C}
\definecolor{arch}{HTML}{1793D1}

\hypersetup{
	colorlinks=true,
	linkcolor=blue,
	urlcolor=cyan,
	citecolor=magenta,
}
%}}}
%Esto controla a la cabecera {{{
\usepackage{fancyhdr}

\pagestyle{fancy}
\fancyhf{}
\renewcommand{\headrulewidth}{0pt}
\chead{ \textbf{\normalsize{Comunicación II} }}
\fancyhf[HL]{\includesvg[height=0.8\headheight]{Utec.svg}}
\fancyfoot[C]{\textbf{\thepage}}
\setlength{\headheight}{52pt}
\setlength{\textheight}{700pt}
%}}}
% Título {{{
\title{\textbf{Creo que una introducción}}
%Aqui hay que poner a los autores
\author{
		Alberto Oporto Ames\\
		\texttt{alberto.oporto@utec.edu.pe}
		}
%}}}
%}}}
%Aquí empieza el documento{{{
\begin{document}
\maketitle
\thispagestyle{fancy}

% Lluvia {{{
\section{Lluvia}
\label{sec:Lluvia}

Algo sobre el software libre

\begin{enumerate}
	\item ¿Cuándo?
		\subitem Desde los 90's, o desde la creación de GNU.
	\item ¿Dónde?
		\subitem En el planeta tierra.
	\item ¿Quiénes?
		\subitem Microsoft, Github, la comunidad GNU/linux, Richard Stallman.
		\subitem Usuarios, desarrolladores, hackers, estudiantes.
\end{enumerate}
% Ideas random {{{
\subsection{Ideas random}%
\label{sub:Ideas random}
\begin{itemize}
	\item GNU
	\item Microsoft
	\item Halloween documents
	\item Embrace, extend, extinguish
	\item Github
	\item Richard Stallman
	\item Google
	\item Privacidad
	\item Hackers
	\item IRC
	\item vim
	\item Sponsors
	\item Arch linux \btw
	\item Tiranía
	\item Tengo preguntas pero no respuestas
	\item No puedo ver el futuro
	\item Creo que debería cambiar un poco el tema.
\end{itemize}
% }}}
% }}}
% Esquema {{{
\section{Esquema original}%
\label{sec:Esquema original}

\begin{enumerate}
	\item ¿Cuál es la peor amenaza para el \textbf{FOOS}?
		\begin{enumerate}
			\item ¿Microsoft realmente ama linux?
			\item ¿Está linux cerca de ser victima del \textbf{Embrace, Extend, Extinguish} de microsoft?
			\item ¿Los \textbf{sponsors} malograrán los proyectos \textbf{FOOS}?
		\end{enumerate}
	\item ¿Algún día linux será \textbf{mainstream}?
		\begin{enumerate}
			\item ¿Alguna vez será el año del escritorio linux?
			\item ¿Los ancianos son capaces de usar linux?
			\item \underline{¿Los \textbf{normies} son capaces de usar linux?}
			\item \underline{¿La comunidad de linux es elitista?}
				{\tiny\color{arch}\texttt{Btw I use arch  }}
				\begin{enumerate}
					\item \underline{¿Linux es solo para nerds?}
					\item \underline{¿Linux es solo para servidores?}
				\end{enumerate}
		\end{enumerate}
	\item ¿Realmente vale la pena no ser parte de la \textbf{botnet}?
		\begin{enumerate}
			\item ¿Debemos confiar en microsoft?
			\item ¿Microsoft espía a los usuarios de windows 10?
			\item ¿Qué hace microsoft con nuestra \textbf{data}?
			\item ¿Cuán \textbf{bloated} es windows 10?
		\end{enumerate}
\end{enumerate}
%}}}
% Esquema 2 {{{
\section{Esquema 2}%
\label{sec:Esquema 2}

\begin{enumerate}
	\item
		\begin{enumerate}
			\item
			\item
			\item
		\end{enumerate}
	\item Quizá
		\begin{enumerate}
			\item Tal vez cuando la comunidad deje de estar tan fragmentada.
				O cuando termine el soporte a windows 7 en el 2020.
			\item Ubuntu
			\item Ubuntu
			\item Sí
				\begin{enumerate}
					\item Ubuntu es para novatos.
					\item También existe el escritorio linux.
				\end{enumerate}
		\end{enumerate}
	\item Si es que te importa tu privacidad
		\begin{enumerate}
			\item Cómo usuario no, como desarrollador sí.
			\item Quién sabe.
			\item Quién sabe x2.
			\item
		\end{enumerate}
\end{enumerate}

% }}}
% Introducción {{{
\section{Introducción}%
\label{sec:Introducción}

Tu, ¿quieres ser libre?
¿Tú computadora se ha vuelto más lenta con el pasar del tiempo?
¿Y sientes que ya no tienes control sobre tu sistema operativo?

El software libre te permitirá escapar de la tiranía del software propietario.
De word a markdown o \LaTeX.
De windows a Linux.
De vivir enjaulado a la libertad.
% }}}
% Desarrollo{{{
\section{Desarrollo}%
\label{sec:Desarrollo}

\subsection{Profundización en la explicación}%

\subsubsection{Hechos/situaciones importantes}%
La creación de Gnu en 1984 por Richard Stallman. \citet[20]{faif}
La compra de Github por Microsoft en el 2018. \citet{AMA}

\subsubsection{Ideas/conceptos relevantes}%

\subsection{Argumentaciones}%

\subsubsection{Postura A}%
Microsoft es una amenaza al software libre.

\subsubsection{Postura B}%

Agumento a1
premisa a.1.1
premisa a.1.2

Agumento a2
premisa a.2.1
premisa a.2.2

% }}}
\end{document}
%}}}
