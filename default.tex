% Proyecto aburrido de comunicación
% Copyright © 2019 Alberto Oporto

% This program is free software: you can redistribute it and/or modify
% it under the terms of the GNU General Public License as published by
% the Free Software Foundation, either version 3 of the License, or
% (at your option) any later version.

% This program is distributed in the hope that it will be useful,
% but WITHOUT ANY WARRANTY; without even the implied warranty of
% MERCHANTABILITY or FITNESS FOR A PARTICULAR PURPOSE.  See the
% GNU General Public License for more details.

% You should have received a copy of the GNU General Public License
% along with this program.  If not, see <http://www.gnu.org/licenses/>.

\documentclass[12pt, twoside]{article}
% Opciones {{{
\usepackage[pdfa, pdfusetitle, unicode=true]{hyperref}
\usepackage[spanish]{babel}
\usepackage[margin=2.5cm, a4paper]{geometry}
\usepackage{luacode}
\usepackage[shortlabels]{enumitem}
\usepackage{import}
\usepackage{xcolor}
\usepackage{fontspec}
\usepackage[mark]{gitinfo2}
\usepackage{setspace}

% Cosas para la bibliografía {{{
\usepackage{csquotes}
\usepackage[natbib,
	backend=biber,
	style=apa,
	apabackref=true
	]{biblatex}
\addbibresource{default.bib}
% }}}
% La profesora me obligó a poner esto {{{
\doublespacing

\addto\captionsspanish{% Sacado de https://tex.stackexchange.com/questions/28516/how-to-change-the-title-of-toc
	\renewcommand{\contentsname}%
	{Esquema}%
}
% }}}

\setmonofont{InconsolataGo Nerd Font}

\newcommand{\btw}{{\color{arch}\texttt{}} }
\newcommand{\git}{{\color{git}\texttt{}} }

\renewcommand{\gitMarkPref}{{\Large\git git}}

%Esto sirve para poner imágenes{{{
\usepackage{graphicx}
\usepackage{svg}
\usepackage{subcaption}

\usepackage{float}
\usepackage{pgfplots}
\usepackage{cancel}

\pgfplotsset{compat=1.16}
\graphicspath{ {ima/} }
%}}}
%Colores de los links {{{
\definecolor{red}{HTML}{F22C40}
\definecolor{green}{HTML}{5AB738}
\definecolor{yellow}{HTML}{D5911A}
\definecolor{blue}{HTML}{407EE7}
\definecolor{magenta}{HTML}{6666EA}
\definecolor{cyan}{HTML}{00AD9C}
\definecolor{arch}{HTML}{1793D1}
\definecolor{git}{HTML}{F54D27}

\hypersetup{
	colorlinks=true,
	linkcolor=blue,
	urlcolor=cyan,
	citecolor=magenta,
}
%}}}
%Esto controla a la cabecera {{{
\usepackage{fancyhdr}

\pagestyle{fancy}
\fancyhf{}
\renewcommand{\headrulewidth}{0pt}
%\chead{ \textbf{\normalsize{Comunicación II} }}
%\fancyhf[HL]{\includesvg[height=0.8\headheight]{Utec.svg}}
%\fancyfoot[EL,OR]{\textbf{\thepage}}
\fancyhead[R]{\textbf{\thepage}}
\fancyhead[L]{ \textbf{FOOS}}
\setlength{\headheight}{40pt}
\setlength{\textheight}{675pt}
%}}}
% Título {{{
\title{\textbf{Microsoft y software libre}}
\date{}
%Aqui hay que poner a los autores
\author{
		Alberto Oporto Ames\\
		%\texttt{alberto.oporto@utec.edu.pe}
		Universidad de Ingeniería y Tecnología (UTEC)
		}
%}}}
%}}}
% Aquí empieza el documento{{{
\begin{document}
% Carátula {{{
\maketitle
\vfill

\noindent Sección: 9\\
Profesora: Oriana Vidal\\
Fecha de entrega: jueves 11 de julio\bigskip\bigskip\\
Artículo académico. Entrega final\\
Laboratorio de comunicación 2\\
Ciclo 2019-1

\thispagestyle{fancy}
\newpage
% }}}
% Esquema {{{

\tableofcontents
\newpage

% }}}
% Cosas antiguas {{{
\iffalse
% Lluvia {{{
\section{Lluvia}
\label{sec:Lluvia}

Algo sobre el software libre

\begin{enumerate}
	\item ¿Cuándo?
		\subitem Desde los 90's, o desde la creación de GNU.
	\item ¿Dónde?
		\subitem En el planeta tierra.
	\item ¿Quiénes?
		\subitem Microsoft, Github, la comunidad GNU/linux, Richard Stallman.
		\subitem Usuarios, desarrolladores, hackers, estudiantes.
\end{enumerate}
% Ideas random {{{
\subsection{Ideas random}%
\label{sub:Ideas random}
\begin{itemize}
	\item GNU
	\item Microsoft
	\item Halloween documents
	\item Embrace, extend, extinguish
	\item Github
	\item Richard Stallman
	\item Google
	\item Privacidad
	\item Hackers
	\item IRC
	\item vim
	\item Sponsors
	\item Arch linux \btw
	\item Tiranía
	\item Tengo preguntas pero no respuestas
	\item No puedo ver el futuro
	\item Creo que debería cambiar un poco el tema.
\end{itemize}
% }}}
% }}}
% Esquema {{{
\section{Esquema original}%
\label{sec:Esquema original}

\begin{enumerate}
	\item ¿Cuál es la peor amenaza para el \textbf{FOOS}?
		\begin{enumerate}
			\item ¿Microsoft realmente ama linux?
			\item ¿Está linux cerca de ser victima del \textbf{Embrace, Extend, Extinguish} de microsoft?
			\item ¿Los \textbf{sponsors} malograrán los proyectos \textbf{FOOS}?
		\end{enumerate}
	\item ¿Algún día linux será \textbf{mainstream}?
		\begin{enumerate}
			\item ¿Alguna vez será el año del escritorio linux?
			\item ¿Los ancianos son capaces de usar linux?
			\item \underline{¿Los \textbf{normies} son capaces de usar linux?}
			\item \underline{¿La comunidad de linux es elitista?}
				{\tiny\color{arch}\texttt{Btw I use arch  }}
				\begin{enumerate}
					\item \underline{¿Linux es solo para nerds?}
					\item \underline{¿Linux es solo para servidores?}
				\end{enumerate}
		\end{enumerate}
	\item ¿Realmente vale la pena no ser parte de la \textbf{botnet}?
		\begin{enumerate}
			\item ¿Debemos confiar en microsoft?
			\item ¿Microsoft espía a los usuarios de windows 10?
			\item ¿Qué hace microsoft con nuestra \textbf{data}?
			\item ¿Cuán \textbf{bloated} es windows 10?
		\end{enumerate}
\end{enumerate}
%}}}
% Esquema 2 {{{
\section{Esquema 2}%
\label{sec:Esquema 2}

\begin{enumerate}
	\item
		\begin{enumerate}
			\item
			\item
			\item
		\end{enumerate}
	\item Quizá
		\begin{enumerate}
			\item Tal vez cuando la comunidad deje de estar tan fragmentada.
				O cuando termine el soporte a windows 7 en el 2020.
			\item Ubuntu
			\item Ubuntu
			\item Sí
				\begin{enumerate}
					\item Ubuntu es para novatos.
					\item También existe el escritorio linux.
				\end{enumerate}
		\end{enumerate}
	\item Si es que te importa tu privacidad
		\begin{enumerate}
			\item Cómo usuario no, como desarrollador sí.
			\item Quién sabe.
			\item Quién sabe x2.
			\item
		\end{enumerate}
\end{enumerate}

% }}}
\fi
% }}}
% Introducción {{{
\section{Introducción}%
\label{sec:Introducción}
\subsection{Decepción}%

Antes que nada, la fuente ``Times New Roman"\ es propietaria \citet{NEWROMAN}.
Me niego a usarla.
Porque esta restringida por una EULA.

Depender del software propietario es una mala idea.
Y obligar a otros a usarlo es aún peor.
Ser libre requiere de sacrificios.
Si no estás dispuesto a hacerlos seguirás siendo víctima del software y no software propietario.

Puede ser que solo sea una fuente, pero aún así siento decepción.
En cambio ``Computer Modern", la fuente que estoy usando, está en el dominio público y \TeX Live, la distribución de \LaTeX
, el lenguage de marcado con el que estoy escribiendo esto, que estoy usando, tiene como licencia la GPL según \citet{TEXLIVE}.

GPL, GNU Public license, esta licencia permite redistribuir, modificar, estudiar y usar software más libremente que las EULAs del software propietario.
Aunque la GPL prohíbe usar programas propietarios con software GPL.
El aislamiento es un pequeño precio por la libertad.

%Tu, ¿quieres ser libre?
%¿Tú computadora se ha vuelto más lenta con el pasar del tiempo?
%¿Y sientes que ya no tienes control sobre tu sistema operativo?
%
%El software libre te permitirá escapar de la tiranía del software propietario.
%De word a markdown o \LaTeX.
%De windows a Linux.
%De vivir alimentando a la botnet a la libertad.
\subsection{GNU}%
\label{sub:GNU}

En 1983 Richard Stallman mandó un mensaje a net.unix-wizards \citet[89]{faif}.
Ese fue el inicio de GNU.
Después, en 1991, Linux Torvalds empezó a crear Linux, un kernel.
GNU y Linux se fusionaron para crear GNU/Linux.
Durante las siguientes dos décadas muchas distribuciones nacieron y murieron.
Algunas siguen existiendo como Debian o Slackware. \citet{DISTROS}

Entonces ahora había un sistema operativo completamente gratis.
Y, ¿Quién se vería afectado por eso?
Microsoft, por la nueva competencia a windows. \citet{HALLO1}

Bueno, en el 2018 Microsoft compró github. \citet{GITHUB}

¿Estará Microsoft intentando recuperar su antiguo monopolio?.
El de la época en la que su CEO dijo que Linux era cancer. \citet{CANCER}

¿O habran cambiado desde que Satya Nadella empezó a ser el CEO?
% }}}
% Plantilla del Desarrollo{{{
\iffalse
\section{Plantilla del desarrollo}%

\subsection{Profundización en la explicación}%

\subsubsection{Hechos/situaciones importantes}%
%La creación de Gnu en 1984 por Richard Stallman. \citet[20]{faif}
%La compra de Github por Microsoft en el 2018. \citet{AMA}
%La creación del Windows subsystem for Linux.
%%El cambio de licensia de algunos programas de Microsoft.
%Microsoft volviendo open source algunos de sus productos.
%Satya Nadella siendo el CEO de Microsoft.

\subsubsection{Ideas/conceptos relevantes}%

%\noindent El software libre es mejor que el software propietario.\\
%El software propietario es mejor que el software libre.

\subsection{Argumentaciones}%

\subsubsection{Postura A}%
%Microsoft es una amenaza al software libre.

Agumento a1\\
premisa a.1.1\\
premisa a.1.2\\
%Tiene una silla en el Linux foundation
%Está intentando entrar al mailing list de seguridad de linux\\
Agumento a2\\
premisa a.2.1\\
premisa a.2.2

\subsubsection{Postura B}%
%Microsoft ha cambiado y ya no es una amenaza.

Agumento a1
premisa a.1.1\\
%Volvió open source algunos de sus programas
premisa a.1.2

Agumento a2
premisa a.2.1
%Ya no tienen al mismo CEO.
premisa a.2.2
\fi
% }}}
% Desarrollo {{{
\section{Desarrollo}%
\label{sec:Desarrollo}
\subsection{Microsoft es una amenaza}%
\label{sub:Microsoft es una amenaza}

La compra podría ser prueba de lo primero.
Como también la distribución del kernel Linux en el WSL2 (Windows subsystem for Linux). \citet{WSL2}
Este permite correr programas de Linux en Windows más rápidamente que una máquina virtual.
Haciendo que los desarrolladores no necesiten salir de Windows y así recuperar a los desarrolladores que ahora usan Linux. \citet{STACK}
Parece el ``Embrace, Extend, Extinguish"\ que menciona \citet{WSL2}.
El ``Embrace", Microsoft volvió open source algunos de sus programas.
El ``Extend", Microsoft compró Github y creo el WSL.
Y el ``Extinguish", la muerte de Linux.
\subsection{Microsoft ha cambiado}%
\label{sub:Microsoft ha cambiado}

Aunque desde la compra; Github no ha sido abandonado, y ahora tiene repositorio privados gratis.
Y también después de que ellos compraran Mojang, los de Minecraft, siguieron sacando ports de Minecraft para plataformas,
como Nintendo Switch, (new) 3ds, Linux (Java) y Playstation, que son la competencia de Xbox y Windows 10.
Además en Azure, la nube de Microsoft, Linux es más usado que el propio Windows server \citet{AZURE}.
Es como si Microsoft necesitase de la comunidad de Linux para poder sobrevivir.
El CEO ya no es el mismo y ya han pasado dos décadas, Linux ya no es tan de nicho como en aquella época.

% }}}
% Cierre {{{
\section{Cierre}%
\label{sec:Cierre}

Hay muchas cosas que no conozco.

Microsoft, aún con el DRM, actualizaciones que no se pueden cancelar, la telemetría en VS code y la ausencia de muchos de sus programas en Linux,
han ayudado al FOOS no dejando morir al repositorio de repositorios, Github.
Ya no estamos en los 90's Microsoft ha cambiado y Linux también.

Pero, ¿Algún día Linux será mainstream?
¿O seguirá siendo algo para los que sean buenos con las computadoras?
\btw
% }}}
\newpage
%\nocite{*}
\printbibliography[heading=bibintoc, title={Lista de referencias}]

\end{document}
%}}}
