\documentclass[12pt]{article}
% Opciones {{{
\usepackage[pdfa, pdfusetitle, unicode=true]{hyperref}
\usepackage[spanish]{babel}
\usepackage[margin=1.5cm, a4paper]{geometry}
\usepackage{luacode}
\usepackage[shortlabels]{enumitem}
\usepackage{import}
\usepackage{xcolor}
\usepackage{fontspec}

% Btw I use arch
\setmonofont{InconsolataGo Nerd Font}
\newcommand{\btw}{{\color{arch}\texttt{ }} }

% Esto sirve para poner ecuaciones
\usepackage{mathtools}

% Esto sirve para poner imágenes{{{
\usepackage{graphicx}
\usepackage{svg}
\usepackage{subcaption}

\usepackage{float}
\usepackage{pgfplots}

\pgfplotsset{compat=1.16}
\graphicspath{ {ima/} }
%}}}
% Colores de los links {{{
\definecolor{red}{HTML}{F22C40}
\definecolor{green}{HTML}{5AB738}
\definecolor{yellow}{HTML}{D5911A}
\definecolor{blue}{HTML}{407EE7}
\definecolor{magenta}{HTML}{6666EA}
\definecolor{cyan}{HTML}{00AD9C}
\definecolor{arch}{HTML}{1793D1}

\hypersetup{
	colorlinks=true,
	linkcolor=blue,
	urlcolor=cyan
}
%}}}
% Esto controla a la cabecera {{{
\usepackage{fancyhdr}

\pagestyle{fancy}
\fancyhf{}
\renewcommand{\headrulewidth}{0pt}
\chead{ \textbf{\normalsize{Comunicación I} }}
\fancyhf[HL]{\includesvg[height=0.8\headheight]{Utec.svg}}
\fancyhf[HR]{{\Huge\btw}}
\fancyfoot[C]{\textbf{\thepage}}
\setlength{\headheight}{52pt}
\setlength{\textheight}{700pt}
%}}}
% Título {{{
\title{\textbf{Guía}}
% Aqui hay que poner a los autores
\author{
		Alberto Oporto Ames\\
		\texttt{alberto.oporto@utec.edu.pe}
		}
%}}}
%}}}
% Aquí empieza el documento{{{
\begin{document}
\maketitle
\thispagestyle{fancy}

%¿Quieres ser libre?
Hace mucho tiempo, cuando las computadoras ocupaban habitaciones, había un hacker.
Su nombre era Richard Stallman.
El trabaja en el laboratorio de inteligencia artificial del MIT.
Programando y hackeando hasta que toda su comunidad se fue del laboratorio.
Por eso y otras cosas más el empezo a crear GNU (GNU's not Unix).

1991:\\
Varios años pasaron y GNU estaba cada vez más completo.
Faltaba una parte muy importante, el kernel.

Mientras tanto, Linus Torvalds estaba creando Linux al inicio como un reemplazo a Minix.
Después GNU y Linux se fusionaron para crear un sistema operativo libre GNU/Linux.

Pasó mucho tiempo y Linux se hizo más popular,
pero aún sigue siendo nada en comparación a windows en el mercado de las computadoras de escritorio.
Aunque ahora hay distribuciones que son amigables para las personas normales.
Como Ubuntu.
Y otras que no lo son.
Arch linux, gentoo, parabola.

2018:\\
Microsoft compró github.

¿Deberíamos confiar nuestros repositorios a Microsoft?
Bueno, últimamente ha estado volviendo open source algunos de sus programas como: visual studio code, .NET, la calculadora de windows, windows terminal.
Y desde la compra de github, este no se ha vuelto un pueblo fantasma y hay repositorios privados gratis.

Aunque también windows 10 es vulnerable a virus,
linux es más usado en el mundo de los servidores,
su sistema operativo para celulares fracasó,
sus libros con drm dejaran de funcionar,
y están intentando entrar al mailing list de seguridad de linux.

\vfill
Repositorio de Github \texttt{:} \url{https://github.com/otreblan/comu2}

\end{document}
%}}}
