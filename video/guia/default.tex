\documentclass[12pt]{article}
% Opciones {{{
\usepackage[pdfa, pdfusetitle, unicode=true]{hyperref}
\usepackage[spanish]{babel}
\usepackage[margin=1.5cm, a4paper]{geometry}
\usepackage{luacode}
\usepackage[shortlabels]{enumitem}
\usepackage{import}
\usepackage{xcolor}
\usepackage{fontspec}
\usepackage[mark]{gitinfo2}

% Btw I use arch
\setmonofont{InconsolataGo Nerd Font}
\newcommand{\btw}{{\color{arch}\texttt{ }} }

% Esto sirve para poner ecuaciones
\usepackage{mathtools}

% Esto sirve para poner imágenes{{{
\usepackage{graphicx}
\usepackage{svg}
\usepackage{subcaption}

\usepackage{float}
\usepackage{pgfplots}

\pgfplotsset{compat=1.16}
\graphicspath{ {ima/} }
%}}}
% Colores de los links {{{
\definecolor{red}{HTML}{F22C40}
\definecolor{green}{HTML}{5AB738}
\definecolor{yellow}{HTML}{D5911A}
\definecolor{blue}{HTML}{407EE7}
\definecolor{magenta}{HTML}{6666EA}
\definecolor{cyan}{HTML}{00AD9C}
\definecolor{arch}{HTML}{1793D1}

\hypersetup{
	colorlinks=true,
	linkcolor=blue,
	urlcolor=cyan
}
%}}}
% Esto controla a la cabecera {{{
\usepackage{fancyhdr}

\pagestyle{fancy}
\fancyhf{}
\renewcommand{\headrulewidth}{0pt}
\chead{ \textbf{\normalsize{Comunicación I} }}
\fancyhf[HL]{\includesvg[height=0.8\headheight]{Utec.svg}}
\fancyhf[HR]{{\Huge\btw}}
\fancyfoot[R]{\textbf{\thepage}}
\setlength{\headheight}{52pt}
\setlength{\textheight}{700pt}
%}}}
% Título {{{
\title{\textbf{Guía}}
% Aqui hay que poner a los autores
\author{
		Alberto Oporto Ames\\
		\texttt{alberto.oporto@utec.edu.pe}
		}
%}}}
%}}}
% Aquí empieza el documento{{{
\begin{document}
\maketitle
\thispagestyle{fancy}

%¿Quieres ser libre?
Algo sobre software libre con dibujos raros.

Hace mucho tiempo, cuando las computadoras ocupaban habitaciones, había un hacker.
Su nombre era Richard Stallman.
El trabajaba en el laboratorio de inteligencia artificial del MIT.
Programando y hackeando hasta que toda su comunidad se fue del laboratorio.
Por eso y otras cosas más el empezó a crear GNU (GNU's not Unix).

1991:\\
Varios años pasaron y GNU estaba cada vez más completo.
Pero aun faltaba una parte muy importante, el kernel.

Mientras tanto, Linus Torvalds estaba creando Linux, al inicio como un reemplazo a Minix.
Poco a poco linux se fue haciendo más popular y empezó a recibir más colaboradores al código fuente.
Nacieron las primeras distribuciones GNU/Linux, al inicio como algo que solo alguien que sea bueno con las computadoras podría usar,
plagado de bugs y con muchas características faltantes.
Y se mantuvo así hasta hace relativamente poco.

Pasó mucho tiempo y Linux se hizo más popular,
pero aún sigue siendo nada en comparación a windows en el mercado de las computadoras de escritorio.
Aunque ahora hay distribuciones que son amigables para las personas normales.
Como Ubuntu.
Y otras que no lo son.
Como Arch linux, gentoo, parabola o Linux from scratch.

2018:\\
Microsoft compró github.

¿Deberíamos confiar nuestros repositorios a Microsoft?
Bueno, últimamente ha estado volviendo open source algunos de sus productos como: visual studio code, .NET, la calculadora de windows, windows terminal.
En Azure, la nube de microsoft, Linux ha superado a Windows.
Y desde la compra de github, este no se ha vuelto un pueblo fantasma y ahora hay repositorios privados gratis.

Aunque también windows 10 es vulnerable a virus,
linux es más usado en el mundo de los servidores,
su sistema operativo para celulares fracasó,
sus libros con drm dejaran de funcionar,
sacaron el windows subsystem for linux que permite correr programas de linux en windows,
y están intentando entrar al mailing list de seguridad de linux.

Al final, ¿Microsoft habrá cambiado desde la época en la que su CEO dijo que linux era cáncer?
o ¿estarán intentando recuperar su monopolio deshaciéndose de linux?

\vfill
Repositorio de Github \texttt{:} \url{https://github.com/otreblan/comu2}

\end{document}
%}}}
